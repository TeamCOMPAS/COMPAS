\documentclass[10pt,a4paper,onecolumn]{article}
\usepackage{marginnote}
\usepackage{graphicx}
\usepackage{xcolor}
\usepackage{authblk,etoolbox}
\usepackage{titlesec}
\usepackage{calc}
\usepackage{tikz}
\usepackage{hyperref}
\hypersetup{colorlinks,breaklinks,
            urlcolor=[rgb]{0.0, 0.5, 1.0},
            linkcolor=[rgb]{0.0, 0.5, 1.0}}
\usepackage{caption}
\usepackage{tcolorbox}
\usepackage{amssymb,amsmath}
\usepackage{ifxetex,ifluatex}
\usepackage{seqsplit}
\usepackage{xstring}

\usepackage{float}
\let\origfigure\figure
\let\endorigfigure\endfigure
\renewenvironment{figure}[1][2] {
    \expandafter\origfigure\expandafter[H]
} {
    \endorigfigure
}

\usepackage{fixltx2e} % provides \textsubscript
\usepackage[
  backend=biber,
%  style=alphabetic,
%  citestyle=numeric
]{biblatex}
\bibliography{paper.bib}

% --- Splitting \texttt --------------------------------------------------

\let\textttOrig=\texttt
\def\texttt#1{\expandafter\textttOrig{\seqsplit{#1}}}
\renewcommand{\seqinsert}{\ifmmode
  \allowbreak
  \else\penalty6000\hspace{0pt plus 0.02em}\fi}


% --- Pandoc does not distinguish between links like [foo](bar) and
% --- [foo](foo) -- a simplistic Markdown model.  However, this is
% --- wrong:  in links like [foo](foo) the text is the url, and must
% --- be split correspondingly.
% --- Here we detect links \href{foo}{foo}, and also links starting
% --- with https://doi.org, and use path-like splitting (but not
% --- escaping!) with these links.
% --- Another vile thing pandoc does is the different escaping of
% --- foo and bar.  This may confound our detection.
% --- This problem we do not try to solve at present, with the exception
% --- of doi-like urls, which we detect correctly.


\makeatletter
\let\href@Orig=\href
\def\href@Urllike#1#2{\href@Orig{#1}{\begingroup
    \def\Url@String{#2}\Url@FormatString
    \endgroup}}
\def\href@Notdoi#1#2{\def\tempa{#1}\def\tempb{#2}%
  \ifx\tempa\tempb\relax\href@Urllike{#1}{#2}\else
  \href@Orig{#1}{#2}\fi}
\def\href#1#2{%
  \IfBeginWith{#1}{https://doi.org}%
  {\href@Urllike{#1}{#2}}{\href@Notdoi{#1}{#2}}}
\makeatother


% --- Page layout -------------------------------------------------------------
\usepackage[top=3.5cm, bottom=3cm, right=1.5cm, left=1.0cm,
            headheight=2.2cm, reversemp, includemp, marginparwidth=4.5cm]{geometry}

% --- Default font ------------------------------------------------------------
% \renewcommand\familydefault{\sfdefault}

% --- Style -------------------------------------------------------------------
\renewcommand{\bibfont}{\small \sffamily}
\renewcommand{\captionfont}{\small\sffamily}
\renewcommand{\captionlabelfont}{\bfseries}

% --- Section/SubSection/SubSubSection ----------------------------------------
\titleformat{\section}
  {\normalfont\sffamily\Large\bfseries}
  {}{0pt}{}
\titleformat{\subsection}
  {\normalfont\sffamily\large\bfseries}
  {}{0pt}{}
\titleformat{\subsubsection}
  {\normalfont\sffamily\bfseries}
  {}{0pt}{}
\titleformat*{\paragraph}
  {\sffamily\normalsize}


% --- Header / Footer ---------------------------------------------------------
\usepackage{fancyhdr}
\pagestyle{fancy}
\fancyhf{}
%\renewcommand{\headrulewidth}{0.50pt}
\renewcommand{\headrulewidth}{0pt}
\fancyhead[L]{\hspace{-0.75cm}\includegraphics[width=5.5cm]{joss-logo.png}}
\fancyhead[C]{}
\fancyhead[R]{}
\renewcommand{\footrulewidth}{0.25pt}

\fancyfoot[L]{\parbox[t]{0.98\headwidth}{\footnotesize{\sffamily Team COMPAS, (2021). COMPAS:
A rapid binary population synthesis
suite. \textit{Journal of Open Source Software}, x(xx), xxx. \url{https://doi.org/xxx}}}}


\fancyfoot[R]{\sffamily \thepage}
\makeatletter
\let\ps@plain\ps@fancy
\fancyheadoffset[L]{4.5cm}
\fancyfootoffset[L]{4.5cm}

% --- Macros ---------

\definecolor{linky}{rgb}{0.0, 0.5, 1.0}

\newtcolorbox{repobox}
   {colback=red, colframe=red!75!black,
     boxrule=0.5pt, arc=2pt, left=6pt, right=6pt, top=3pt, bottom=3pt}

\newcommand{\ExternalLink}{%
   \tikz[x=1.2ex, y=1.2ex, baseline=-0.05ex]{%
       \begin{scope}[x=1ex, y=1ex]
           \clip (-0.1,-0.1)
               --++ (-0, 1.2)
               --++ (0.6, 0)
               --++ (0, -0.6)
               --++ (0.6, 0)
               --++ (0, -1);
           \path[draw,
               line width = 0.5,
               rounded corners=0.5]
               (0,0) rectangle (1,1);
       \end{scope}
       \path[draw, line width = 0.5] (0.5, 0.5)
           -- (1, 1);
       \path[draw, line width = 0.5] (0.6, 1)
           -- (1, 1) -- (1, 0.6);
       }
   }

% --- Title / Authors ---------------------------------------------------------
% patch \maketitle so that it doesn't center
\patchcmd{\@maketitle}{center}{flushleft}{}{}
\patchcmd{\@maketitle}{center}{flushleft}{}{}
% patch \maketitle so that the font size for the title is normal
\patchcmd{\@maketitle}{\LARGE}{\LARGE\sffamily}{}{}
% patch the patch by authblk so that the author block is flush left
\def\maketitle{{%
  \renewenvironment{tabular}[2][]
    {\begin{flushleft}}
    {\end{flushleft}}
  \AB@maketitle}}
\makeatletter
\renewcommand\AB@affilsepx{ \protect\Affilfont}
%\renewcommand\AB@affilnote[1]{{\bfseries #1}\hspace{2pt}}
\renewcommand\AB@affilnote[1]{{\bfseries #1}\hspace{3pt}}
\renewcommand{\affil}[2][]%
   {\newaffiltrue\let\AB@blk@and\AB@pand
      \if\relax#1\relax\def\AB@note{\AB@thenote}\else\def\AB@note{#1}%
        \setcounter{Maxaffil}{0}\fi
        \begingroup
        \let\href=\href@Orig
        \let\texttt=\textttOrig
        \let\protect\@unexpandable@protect
        \def\thanks{\protect\thanks}\def\footnote{\protect\footnote}%
        \@temptokena=\expandafter{\AB@authors}%
        {\def\\{\protect\\\protect\Affilfont}\xdef\AB@temp{#2}}%
         \xdef\AB@authors{\the\@temptokena\AB@las\AB@au@str
         \protect\\[\affilsep]\protect\Affilfont\AB@temp}%
         \gdef\AB@las{}\gdef\AB@au@str{}%
        {\def\\{, \ignorespaces}\xdef\AB@temp{#2}}%
        \@temptokena=\expandafter{\AB@affillist}%
        \xdef\AB@affillist{\the\@temptokena \AB@affilsep
          \AB@affilnote{\AB@note}\protect\Affilfont\AB@temp}%
      \endgroup
       \let\AB@affilsep\AB@affilsepx
}
\makeatother
\renewcommand\Authfont{\sffamily\bfseries}
\renewcommand\Affilfont{\sffamily\small\mdseries}
\setlength{\affilsep}{1em}


\ifnum 0\ifxetex 1\fi\ifluatex 1\fi=0 % if pdftex
  \usepackage[T1]{fontenc}
  \usepackage[utf8]{inputenc}

\else % if luatex or xelatex
  \ifxetex
    \usepackage{mathspec}
  \else
    \usepackage{fontspec}
  \fi
  \defaultfontfeatures{Ligatures=TeX,Scale=MatchLowercase}

\fi
% use upquote if available, for straight quotes in verbatim environments
\IfFileExists{upquote.sty}{\usepackage{upquote}}{}
% use microtype if available
\IfFileExists{microtype.sty}{%
\usepackage{microtype}
\UseMicrotypeSet[protrusion]{basicmath} % disable protrusion for tt fonts
}{}

\usepackage{hyperref}
\hypersetup{unicode=true,
            pdftitle={COMPAS: A rapid binary population synthesis suite},
            pdfborder={0 0 0},
            breaklinks=true}
\urlstyle{same}  % don't use monospace font for urls

% --- We redefined \texttt, but in sections and captions we want the
% --- old definition
\let\addcontentslineOrig=\addcontentsline
\def\addcontentsline#1#2#3{\bgroup
  \let\texttt=\textttOrig\addcontentslineOrig{#1}{#2}{#3}\egroup}
\let\markbothOrig\markboth
\def\markboth#1#2{\bgroup
  \let\texttt=\textttOrig\markbothOrig{#1}{#2}\egroup}
\let\markrightOrig\markright
\def\markright#1{\bgroup
  \let\texttt=\textttOrig\markrightOrig{#1}\egroup}


\usepackage{graphicx,grffile}
\makeatletter
\def\maxwidth{\ifdim\Gin@nat@width>\linewidth\linewidth\else\Gin@nat@width\fi}
\def\maxheight{\ifdim\Gin@nat@height>\textheight\textheight\else\Gin@nat@height\fi}
\makeatother
% Scale images if necessary, so that they will not overflow the page
% margins by default, and it is still possible to overwrite the defaults
% using explicit options in \includegraphics[width, height, ...]{}
\setkeys{Gin}{width=\maxwidth,height=\maxheight,keepaspectratio}
\IfFileExists{parskip.sty}{%
\usepackage{parskip}
}{% else
\setlength{\parindent}{0pt}
\setlength{\parskip}{6pt plus 2pt minus 1pt}
}
\setlength{\emergencystretch}{3em}  % prevent overfull lines
\providecommand{\tightlist}{%
  \setlength{\itemsep}{0pt}\setlength{\parskip}{0pt}}
\setcounter{secnumdepth}{0}
% Redefines (sub)paragraphs to behave more like sections
\ifx\paragraph\undefined\else
\let\oldparagraph\paragraph
\renewcommand{\paragraph}[1]{\oldparagraph{#1}\mbox{}}
\fi
\ifx\subparagraph\undefined\else
\let\oldsubparagraph\subparagraph
\renewcommand{\subparagraph}[1]{\oldsubparagraph{#1}\mbox{}}
\fi

\title{COMPAS: A rapid binary population synthesis suite}

        \author[1]{Team COMPAS}
          \author[2, 3]{Jeff Riley}
          \author[4, 3]{Poojan Agrawal}
          \author[5]{Jim W. Barrett}
          \author[6]{Kristian N. K. Boyett}
          \author[7]{Floor S. Broekgaarden}
          \author[8, 4, 3]{Debatri Chattopadhyay}
          \author[9]{Sebastian M. Gaebel}
          \author[10]{Fabian Gittins}
          \author[2,3]{Ryosuke Hirai}
          \author[11]{George Howitt}
          \author[12, 13, 14]{Stephen Justham}
          \author[12]{Lokesh Khandelwal}
          \author[12]{Floris Kummer}
          \author[2,3]{Mike Y. M. Lau}
          \author[2,3,5]{Ilya Mandel}
          \author[14, 12, 7]{Selma E. de Mink}
          \author[5,3]{Coenraad Neijssel}
          \author[2,3]{Tim Riley}
          \author[7, 12, 14]{Lieke van Son}
          \author[4,3]{Simon Stevenson}
          \author[15, 16]{Alejandro Vigna-Gómez}
          \author[12]{Serena Vinciguerra}
          \author[7, 14]{Tom Wagg}
          \author[2,3]{Reinhold Willcox}
    
      \affil[1]{The public COMPAS code is a product of work by the
entire COMPAS collaboration over many years; we therefore kindly request
that, in recognition of this team effort, the paper is cited as Team
COMPAS - J. Riley et al.}
      \affil[2]{School of Physics and Astronomy, Monash University,
Clayton, Victoria 3800, Australia}
      \affil[3]{OzGrav, Australian Research Council Centre of Excellence
for Gravitational Wave Discovery, Australia}
      \affil[4]{Centre for Astrophysics and Supercomputing, Swinburne
University of Technology, Hawthorn, VIC 3122, Australia}
      \affil[5]{Institute of Gravitational Wave Astronomy and School of
Physics and Astronomy, University of Birmingham, Birmingham, B15 2TT}
      \affil[6]{Department of Physics, University of Oxford, Denys
Wilkinson Building, Keble Road, Oxford OX1 3RH, UK}
      \affil[7]{Center for Astrophysics \textbar Harvard \& Smithsonian,
60 Garden St., Cambridge, MA 02138, USA}
      \affil[8]{School of Physics and Astronomy, Cardiff University,
Cardiff, CF24 3AA, United Kingdom}
      \affil[9]{Max Planck Institute for Gravitational Physics (Albert
Einstein Institute), Callinstrasse 38, D-30167 Hannover, Germany}
      \affil[10]{Mathematical Sciences and STAG Research Centre,
University of Southampton, Southampton SO17 1BJ, UK}
      \affil[11]{School of Physics, University of Melbourne, Parkville,
Victoria, 3010, Australia}
      \affil[12]{Anton Pannekoek Institute of Astronomy and GRAPPA,
Science Park 904, University of Amsterdam, 1098XH Amsterdam, The
Netherlands}
      \affil[13]{School of Astronomy \& Space Science, University of the
Chinese Academy of Sciences, Beijing 100012, China}
      \affil[14]{Max Planck Institute for Astrophysics,
Karl-Schwarzschild-Str. 1, 85748 Garching, Germany}
      \affil[15]{DARK, Niels Bohr Institute, University of Copenhagen,
Jagtvej 128, 2200, Copenhagen, Denmark}
      \affil[16]{Niels Bohr International Academy, The Niels Bohr
Institute, Blegdamsvej 17, 2100 Copenhagen, Denmark}
  \date{\vspace{-5ex}}

\begin{document}
\maketitle

\marginpar{
  \sffamily\small

  {\bfseries DOI:} \href{https://doi.org/xxx}{\color{linky}{xxx}}

  \vspace{2mm}

  {\bfseries Software}
  \begin{itemize}
    \setlength\itemsep{0em}
    \item \href{https://github.com/openjournals/joss-reviews/issues/xxx}{\color{linky}{Review}} \ExternalLink
    \item \href{https://github.com/TeamCOMPAS/COMPAS}{\color{linky}{Repository}} \ExternalLink
    \item \href{http://arXiv.org/abs/2109.10352}{\color{linky}{Archive}} \ExternalLink
  \end{itemize}

  \vspace{2mm}

  {\bfseries Submitted:} XX Month 2021\\
  {\bfseries Published:} XX Month 2021

  \vspace{2mm}
  {\bfseries License}\\
  Authors of papers retain copyright and release the work under a Creative Commons Attribution 4.0 International License (\href{https://creativecommons.org/licenses/by/4.0/}{\color{linky}{CC BY 4.0}}).
}

\hypertarget{summary}{%
\section{Summary}\label{summary}}

Most massive stars---those with initial masses greater than 8
\(M_\odot\)---are born with another similarly massive star as a
companion {[}@Sana:2012Sci{]}. Massive binary stars are responsible for
producing many exotic astrophysical phenomena, such as the observed
diversity of supernovae, binary pulsars, X-ray binaries and merging
compact objects. The latter are now regularly observed by the
ground-based gravitational wave observatories Advanced LIGO and Virgo
{[}@abbott2016observation;@GWTC2{]}. Population models of massive binary
evolution make it possible to interpret existing observations and to
make predictions for future observing campaigns.

\hypertarget{statement-of-need}{%
\section{Statement of need}\label{statement-of-need}}

Binary population synthesis generates population models of isolated
stellar binaries under a set of parametrized assumptions. These models
permit comparisons against observational data sets, such as X-ray
binaries of gravitational-wave mergers.

In particular, rapid binary population synthesis is needed in order to
efficiently explore a broad parameter space of uncertain assumptions
about the physics of stellar and binary evolution, including supernova
remnant masses and natal kicks, mass transfer efficiency and stability,
and the outcome of common-envelope events.

\href{https://compas.science}{COMPAS} is such a rapid binary population
synthesis suite. It parametrizes complex astrophysical processes with
prescriptions calibrated to detailed models. COMPAS is designed to allow
for flexible modifications as evolutionary models improve.

\hypertarget{details}{%
\section{Details}\label{details}}

The core engine of COMPAS---responsible for calculating the evolution of
single {[}@Hurley:2000pk{]} and binary {[}@Hurley:2002rf{]} stars---is
written in object oriented C++ for speed and flexibility. COMPAS is able
to simulate the evolution of a typical binary over 10 Gyr in
approximately 10 milliseconds.

A detailed description of the implementation of the COMPAS suite can be
found in @COMPAS:2021methodsPaper.

In addition to the core stellar and binary evolution engine, we provide
Python scripts for both pre- and post-processing COMPAS outputs.
Post-processing can account for integrating populations formed
throughout cosmic history {[}@2019MNRAS.490.3740N{]} and methods to
account for gravitational-wave selection effects {[}@Barrett:2017fcw{]}.
A set of examples is also provided.

COMPAS is \emph{embarrassingly} parallel and can be trivially run on
high performance computers and distributed on cloud computing.

COMPAS was initially designed to focus on studies of merging binaries
containing neutron stars and black holes that are being observed through
gravitational waves
{[}@Stevenson2017FormationEvolution;@2018MNRAS.481.4009V{]}. In recent
years, the scope of systems investigated with COMPAS has expanded to
incorporate, e.g., Be X-ray binaries {[}@Vinciguerra:2020{]} and
luminous red novae {[}@Howitt:2020{]} (see @COMPAS:2021methodsPaper or
\href{https://compas.science}{the COMPAS collaboration website} for a
summary of COMPAS publications to date.)

COMPAS development happens on
\href{https://github.com/TeamCOMPAS/COMPAS}{Github}. We maintain a
\href{https://zenodo.org/communities/compas/}{Zenodo community} where
data from many publications using COMPAS is publicly available.

\hypertarget{acknowledgements}{%
\section{Acknowledgements}\label{acknowledgements}}

Multiple authors are supported by the Australian Research Council Centre
of Excellence for Gravitational Wave Discovery (OzGrav), through project
number CE170100004. Multiple authors were funded in part by the National
Science Foundation under Grant No.~(NSF grant number 2009131), the
Netherlands Organization for Scientific Research (NWO) as part of the
Vidi research program BinWaves with project number 639.042.728 and by
the European Union's Horizon 2020 research and innovation program from
the European Research Council (ERC, Grant agreement No.~715063). FSB is
supported in part by the Prins Bernard Cultuurfonds studiebeurs. IM is a
recipient of an Australian Research Council Future Fellowship
(FT190100574). AVG acknowledges funding support by the Danish National
Research Foundation (DNRF132)

\hypertarget{references}{%
\section{References}\label{references}}

\end{document}
